%%%%     Please only edit bits which say to edit them     %%%%
%%%% They usually start with %///% to show where they are %%%%


%%% START : DO NOT EDIT %%%%

% Document setup
\documentclass[14pt]{extarticle}

\usepackage{graphicx,geometry,hyperref,enumitem,nicefrac,extsizes,color,nameref,calc,textcomp,csquotes,setspace,titlesec}

\hypersetup{
    pdfborder={0 0 0}
}

% Important: This must come near the start of the setup file

\makeatletter
\newcommand*{\currentname}{\@currentlabelname}
\makeatother

\makeatletter
\def\normaljustify{%
  \let\\\@centercr\rightskip\z@skip \leftskip\z@skip%
  \parfillskip=0pt plus 1fil}
\makeatother

\setlength\parindent{0pt}
\setlength{\parskip}{\baselineskip}
\setcounter{secnumdepth}{2}
\setlength{\fboxsep}{0.4cm}
\setlength{\fboxrule}{0.04cm}

\definecolor{niceblue}{RGB}{166,218,255}
\definecolor{nicebluedark}{RGB}{0,104,179}
\definecolor{nicegreen}{RGB}{202,255,166}
\definecolor{nicegreendark}{RGB}{71,179,0}
\definecolor{nicered}{RGB}{255,166,166}
\definecolor{nicereddark}{RGB}{179,0,0}
\definecolor{niceyellow}{RGB}{255,254,166}
\definecolor{niceyellowdark}{RGB}{219,216,0}

\newcommand{\sectionbreak}{\clearpage}

\newcommand{\actioncol}{niceblue}
\newcommand{\notecol}{nicered}
\newcommand{\questioncol}{niceyellow}
\newcommand{\goalcol}{nicegreen}

\newcommand{\group}[1]{
    \begin{center}
    \parbox[c]{\linewidth} {
    	#1
    }
    \end{center}
}

\newcommand{\colourbox}[2]{
    \begin{center}
    \fcolorbox{#1dark}{#1}{
    	\parbox[c]{.9\linewidth} {
    	    #2 
    	}
    }
    \end{center}
}

\newcommand{\numberlist}[2][]{
    \begin{enumerate}[series=#1]
    	#2
    \end{enumerate}
    \vspace{-\topsep}
}

\newcommand{\resumenumberlist}[2]{
    \begin{enumerate}[resume=#1]
    	#2
    \end{enumerate}
    \vspace{-\topsep}
}

\newcommand{\itemlist}[1]{
    \begin{itemize}
    	#1
    \end{itemize}
    \vspace{-\topsep}
}

\newcommand{\valuelist}[2]{
    \begin{description}[align=left,labelwidth=*,leftmargin=#1]
    	#2
    \end{description}
}

\newcommand{\actions}[2][\currentname acs]{
    \colourbox{\actioncol}{ 
    	{\small\color{\actioncol dark} Actions}
    	\resumenumberlist{ #1 }{ #2 }
    }
}

\newcommand{\restartactions}[2][\currentname acs]{
    \colourbox{\actioncol}{ 
    	{\small\color{\actioncol dark} Actions}
    	\numberlist[#1]{ #2 } 
    }
}

\newcommand{\notes}[1]{
    \colourbox{\notecol}{ 
    	{\small\color{\notecol dark} Notes}\par
    	#1
    }
}

\newcommand{\questions}[2][questions]{
    \colourbox{\questioncol}{ 
    	{\small\color{\questioncol dark} Questions}
    	\resumenumberlist{ #1 }{ #2 } 
    }
}
\newcommand{\restartquestions}[2][questions]{
    \colourbox{\questioncol}{ 
    	{\small\color{\questioncol dark} Questions}
    	\numberlist[#1]{ #2 } 
    }
}
\newcommand{\goals}[1]{
    \colourbox{\goalcol}{ 
    	{\small\color{\goalcol dark} Goals}\par
    	#1 
    }
}

\newcommand{\boxesdescription}{
    \vspace{-\topsep}
    \begin{description}[align=left,labelwidth=*,leftmargin=3cm]
    	\item [{\color{\actioncol dark}Actions}] Stuff for you to do. They are highlighted in blue.
    	
    	\item [{\color{\notecol dark}Notes}] Notes about important stuff you need to be aware of (and possibly remember!). They are highlighted in red.
    	
    	\item [{\color{\questioncol dark} Questions}] Questions you should try to answer. Sometimes you'll need to write things down; other times you'll need to build something in the game. They are highlighted in yellow.

    	{\bfseries Ask a helper or the teacher to check your answers.}
    	
    	\item [{\color{\goalcol dark} Goals}] Stuff you should have completed at the end of each section. They are highlighted in green.
    \end{description}
}

\newcommand{\screenshot}[3][.9\linewidth-0.8cm]{
    \group {
        \begin{center}
            \includegraphics[width=#1]{img/screenshots/#2}\par
            {\small #3}
        \end{center}
    }
}

\newcommand{\image}[3][.9\linewidth-0.8cm]{
    \group {
        \begin{center}
            \includegraphics[width=#1]{img/other/#2}\par
            {\small #3}
        \end{center}
    }
}

\newcommand{\todo}[1]{
    \image[7cm]{todo}{#1}
}

\newcommand{\spacer}{
    \par\vspace{14pt}
}


%%% END : DO NOT EDIT %%%%



\def\workshoptitle{What Is Git?}
\def\workshopsubtitle{And how do I use it?}
\def\workshopauthor{Ed Nutting}



%%% START : DO NOT EDIT %%%%

\begin{document}

\sffamily

% Title page - do not edit
\begin{titlepage}

	\newgeometry{margin=2cm}
	
	\begin{figure}
	\centering
	\begin{minipage}{.5\textwidth}
		\centering
		\includegraphics[width=.6\linewidth]{img/Bristol-University-Logo}
	\end{minipage}%
	\begin{minipage}{.5\textwidth}
		\centering
		\includegraphics[width=.4\linewidth]{img/Digimakers}
	\end{minipage}
	
	\vspace{18pt}
	
	\end{figure}
	
	\centering
	\normalfont
	{\Large Merchant Venturers School of Engineering\par Outreach Programme\par}
	\vspace{2cm}
	\sffamily
	{\huge\bfseries\workshoptitle\par}
	\vspace{0.5cm}
	{\LARGE\bfseries\workshopsubtitle}
	
	\vfill
	
	{\normalsize Created by\par
	\large\slshape\workshopauthor}
	
	\vfill
	
	{\normalsize Organised by\par
	\large\href{mailto:Caroline.Higgins@bristol.ac.uk}{Caroline.Higgins@bristol.ac.uk}}

	\vfill

	{\large Published on \today\par}
\end{titlepage}

% Document setup - do not edit
\newgeometry{margin=1.5cm}

\renewcommand{\abstractname}{Notes to the reader}
\begin{abstract}

\vspace{-\topsep}
    
\noindent
\vspace{-\topsep}
    
\setlist{leftmargin=0mm}
\itemlist {
    \setlength{\itemsep}{18pt}
    
%%% END : DO NOT EDIT %%%%

%///% Begin editable content

    \item This workshop is intended to last 1 hour.
        
    \item This workshop is intended for people with no prior knowledge of Git, GitHub or GitKraken. You don't need to be able to use the command line for this tutorial (or to use Git!).
    
    \item This workshop aims to teach the skills needed for working on small Outreach projects (i.e. other workshops). As such, we will ignore many of the feature of Git that you would need in industry. This is a good starting point though.
    
    \item The content is intended to be learnt through self-directed individual learning by following the worksheet and practicing actions repeatedly until you remember how to do them. 

    \item The learning platform is Git for Windows or Linux and GitHub. For both platforms, we will use the GitKraken user interface (sorry, but command line people can probably already figure out how to use Git on their own).
    
    \item This workshop probably works on Mac but is untested (unless anyone fancies donating a Mac to me? :) \#worthatry )
    
    \item There are many of versions of Git, all of which should be compatible with this workshop (but might not be - if you follow this guide, you'll end up with the right one installed.)

    \item You should already be comfortable using your platform of choice including: installing programs, opening programs, managing files and folders, signing up for stuff via email and accessing the internet (which is presumably how you got this worksheet in the first place?)

    \item This workshop teaches the following skills:
    \itemlist {
        \item Creating an account and repository on Github
        \item Cloning a new/empty repository
        \item Configuring a new/empty repository
        \item Installing a git client
        \item Configuring a git client
        \item Adding (/creating) files inside your project
        \item Pushing and pulling files
        \item Viewing the commit history
        \item Correcting a mistake in a file
        \item Discarding some or all changes to a file
        \item Viewing, creating and commenting-on issues on GitHub
    }
    
% End editable content

%%% START : DO NOT EDIT %%%%

}

\end{abstract}

% Document setup - do not edit
\newgeometry{margin=1.5cm}


%%% END : DO NOT EDIT %%%%

% Sections from this point onwards are your choice, though Introduction, Conclusion (/Wrap-up/What we learnt) and Extra Resources sections are recommended.

%///% Begin editable content

\section{Introduction}

Hi! In this short workshop we're going to try to introduce you to Git - industry's (pretty-much) standard tool for version control and what we use in the MVSE Outreach programme for managing files. Don't worry if you have no idea what \enquote{Version Control} is just yet - we're going to start from the ground up.

% Leave this here on all worksheets, please stick to the format - the aim is for consistency not lots of special fancy stuff on every sheet (though odd bits of fancy stuff is cool).
Let's get started. Each section is made up of four parts:
\boxesdescription

We'll also write some information between parts and include plenty of screenshots to help you out.

\subsection{What is version control?}

TODO


\subsection{Why do we need version control software?}

TODO


\subsection{So what is Git?}

TODO


\subsection{I've heard a lot about the command line...}

TODO


\subsection{What is a repository (a.k.a. repo)?}

TODO


\subsection{So what is GitHub?}

TODO


\subsection{Remote vs. local versions}

TODO


\subsection{And what about GitKraken?}

TODO





\section{Setup GitHub}

\subsection{Creating an account}

TODO


\subsection{Joining the MVSE Outreach group}

TODO


\subsection{Creating a repository}

TODO


\subsubsection{Licenses}

TODO


\subsection{Configuring a repository}

TODO


\subsubsection{Read-Mes}




\section{Setup GitKraken}

\subsection{Re-cap: What is GitKraken?}

TODO


\subsection{Installing GitKraken}

TODO


\subsection{Configuring GitKraken}

TODO





\section{Use GitKraken}

\subsection{Cloning a repository}

TODO


\subsection{Initialising a repository}

TODO


\subsubsection{Git Ignore: .gitignore}

TODO


\subsection{Staging files}

TODO


\subsection{Committing files}

TODO


\subsection{Pushing commits(/changes)}

TODO


\subsection{Viewing the commit history}

TODO


\subsection{Reverting changes}

TODO


\subsection{Discarding changes before committing}

TODO


\subsection{Amending commits}

TODO






\section{Use GitHub}

\subsection{Viewing your repository}

TODO


\subsection{Viewing the commit history}

TODO


\subsection{Viewing issues (Part 1)}

TODO


\subsection{Creating an issue}

TODO


\subsection{Viewing issues (Part 2)}

TODO


\subsection{Commenting on an issue}

TODO


\subsection{Closing an issue}

TODO






\section{Wrap-up}

We hope you enjoyed this workshop! If you have any questions or run into problems, just ask one of the other outreach people - I'm sure they'll be happy to help.

\goals{
    Hmmm...
    \itemlist {
    	\item TODO
    }
}




\section{Extra Resources}

Here's a few extra resources to help you along with this worksheet and some stuff to try later.

\itemlist {
    \item \href{http://www.google.com}{Google is your friend : http://www.google.com}
}

\end{document}
