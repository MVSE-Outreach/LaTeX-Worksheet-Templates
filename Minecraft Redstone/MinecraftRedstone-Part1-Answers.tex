%%%%     Please only edit bits which say to edit them     %%%%
%%%% They usually start with %///% to show where they are %%%%



%%% START : DO NOT EDIT %%%%

% Document setup
\documentclass[14pt]{extarticle}

\usepackage{graphicx,geometry,hyperref,enumitem,nicefrac,extsizes,color,nameref,calc,textcomp,csquotes,setspace,titlesec}

\hypersetup{
    pdfborder={0 0 0}
}

% Important: This must come near the start of the setup file

\makeatletter
\newcommand*{\currentname}{\@currentlabelname}
\makeatother

\makeatletter
\def\normaljustify{%
  \let\\\@centercr\rightskip\z@skip \leftskip\z@skip%
  \parfillskip=0pt plus 1fil}
\makeatother

\setlength\parindent{0pt}
\setlength{\parskip}{\baselineskip}
\setcounter{secnumdepth}{2}
\setlength{\fboxsep}{0.4cm}
\setlength{\fboxrule}{0.04cm}

\definecolor{niceblue}{RGB}{166,218,255}
\definecolor{nicebluedark}{RGB}{0,104,179}
\definecolor{nicegreen}{RGB}{202,255,166}
\definecolor{nicegreendark}{RGB}{71,179,0}
\definecolor{nicered}{RGB}{255,166,166}
\definecolor{nicereddark}{RGB}{179,0,0}
\definecolor{niceyellow}{RGB}{255,254,166}
\definecolor{niceyellowdark}{RGB}{219,216,0}

\newcommand{\sectionbreak}{\clearpage}

\newcommand{\actioncol}{niceblue}
\newcommand{\notecol}{nicered}
\newcommand{\questioncol}{niceyellow}
\newcommand{\goalcol}{nicegreen}

\newcommand{\group}[1]{
    \begin{center}
    \parbox[c]{\linewidth} {
    	#1
    }
    \end{center}
}

\newcommand{\colourbox}[2]{
    \begin{center}
    \fcolorbox{#1dark}{#1}{
    	\parbox[c]{.9\linewidth} {
    	    #2 
    	}
    }
    \end{center}
}

\newcommand{\numberlist}[2][]{
    \begin{enumerate}[series=#1]
    	#2
    \end{enumerate}
    \vspace{-\topsep}
}

\newcommand{\resumenumberlist}[2]{
    \begin{enumerate}[resume=#1]
    	#2
    \end{enumerate}
    \vspace{-\topsep}
}

\newcommand{\itemlist}[1]{
    \begin{itemize}
    	#1
    \end{itemize}
    \vspace{-\topsep}
}

\newcommand{\valuelist}[2]{
    \begin{description}[align=left,labelwidth=*,leftmargin=#1]
    	#2
    \end{description}
}

\newcommand{\actions}[2][\currentname acs]{
    \colourbox{\actioncol}{ 
    	{\small\color{\actioncol dark} Actions}
    	\resumenumberlist{ #1 }{ #2 }
    }
}

\newcommand{\restartactions}[2][\currentname acs]{
    \colourbox{\actioncol}{ 
    	{\small\color{\actioncol dark} Actions}
    	\numberlist[#1]{ #2 } 
    }
}

\newcommand{\notes}[1]{
    \colourbox{\notecol}{ 
    	{\small\color{\notecol dark} Notes}\par
    	#1
    }
}

\newcommand{\questions}[2][questions]{
    \colourbox{\questioncol}{ 
    	{\small\color{\questioncol dark} Questions}
    	\resumenumberlist{ #1 }{ #2 } 
    }
}
\newcommand{\restartquestions}[2][questions]{
    \colourbox{\questioncol}{ 
    	{\small\color{\questioncol dark} Questions}
    	\numberlist[#1]{ #2 } 
    }
}
\newcommand{\goals}[1]{
    \colourbox{\goalcol}{ 
    	{\small\color{\goalcol dark} Goals}\par
    	#1 
    }
}

\newcommand{\boxesdescription}{
    \vspace{-\topsep}
    \begin{description}[align=left,labelwidth=*,leftmargin=3cm]
    	\item [{\color{\actioncol dark}Actions}] Stuff for you to do. They are highlighted in blue.
    	
    	\item [{\color{\notecol dark}Notes}] Notes about important stuff you need to be aware of (and possibly remember!). They are highlighted in red.
    	
    	\item [{\color{\questioncol dark} Questions}] Questions you should try to answer. Sometimes you'll need to write things down; other times you'll need to build something in the game. They are highlighted in yellow.

    	{\bfseries Ask a helper or the teacher to check your answers.}
    	
    	\item [{\color{\goalcol dark} Goals}] Stuff you should have completed at the end of each section. They are highlighted in green.
    \end{description}
}

\newcommand{\screenshot}[3][.9\linewidth-0.8cm]{
    \group {
        \begin{center}
            \includegraphics[width=#1]{img/screenshots/#2}\par
            {\small #3}
        \end{center}
    }
}

\newcommand{\image}[3][.9\linewidth-0.8cm]{
    \group {
        \begin{center}
            \includegraphics[width=#1]{img/other/#2}\par
            {\small #3}
        \end{center}
    }
}

\newcommand{\todo}[1]{
    \image[7cm]{todo}{#1}
}

\newcommand{\spacer}{
    \par\vspace{14pt}
}


%%% END : DO NOT EDIT %%%%



%///% Workshop info - edit these values for your workshop
\def\workshoptitle{Minecraft Redstone}
\def\workshopsubtitle{Part 1 of 2: Answers}
\def\workshopauthor{Ed Nutting}



%%% START : DO NOT EDIT %%%%

\begin{document}

\sffamily
\nosectionbreak

% Title page - do not edit
\begin{titlepage}

	\newgeometry{margin=2cm}
	
	\begin{figure}
	\centering
	\begin{minipage}{.5\textwidth}
		\centering
		\includegraphics[width=.6\linewidth]{img/Bristol-University-Logo}
	\end{minipage}%
	\begin{minipage}{.5\textwidth}
		\centering
		\includegraphics[width=.4\linewidth]{img/Digimakers}
	\end{minipage}
	
	\vspace{18pt}
	
	\end{figure}
	
	\centering
	\normalfont
	{\Large Merchant Venturers School of Engineering\par Outreach Programme\par}
	\vspace{2cm}
	\sffamily
	{\huge\bfseries\workshoptitle\par}
	\vspace{0.5cm}
	{\LARGE\bfseries\workshopsubtitle}
	
	\vfill
	
	{\normalsize Created by\par
	\large\slshape\workshopauthor}
	
	\vfill
	
	{\normalsize Organised by\par
	\large\href{mailto:Caroline.Higgins@bristol.ac.uk}{Caroline.Higgins@bristol.ac.uk}}

	\vfill

	{\large Published on \today\par}
\end{titlepage}

% Document setup - do not edit
\newgeometry{margin=1.5cm}


%%% END : DO NOT EDIT %%%%

% Sections from this point onwards are your choice, though Introduction, Conclusion (/Wrap-up/What we learnt) and Extra Resources sections are recommended.

%///% Begin editable content

\section{Introduction}

\questions{
    \item What kind of block is the Redstone ready world made from?
    
    \answer{Sandstone}
    \item How many blocks vertically downwards are there till you reach the bedrock?
    
    \answer{55 (may vary if inside certain structures)}
}





\section{Placing and Powering Redstone}

\subsection{Boosting power}

\questions {
    \item Which side of a repeater accepts power in?
    
    \answer{Side with the slot / slider.}
    
    \item Which side of a repeater outputs power?
    
    \answer{Side with the fixed-position stick.}
    
    \item How far (counted in blocks) does power travel out of a repeater?
    
    \answer{16}
}



\section{Redstone Torches}

\subsection{Torch on a block}

\questions {
    \item What would happen if you powered a block using a wire from a torch placed on the same block?
    
    \answer{Feedback loop causing rapid on/off switching. Eventually the torch burns out i.e. stops working.}
    
    \item When you power on the wire, is there a delay before the torch on the block switches off?
    
    \answer{Yes - 1 tick}
}

\subsection{NOT and NOR gates}

\questions {
    \item What happens when you power the input of the first NOT gate?
    
    \answer{Feedback loops causes flickering. Eventually, the torch \enquote{burns out} i.e. stops working.}
    
    \item Is two NOT gates in a row like this, the same as a repeater?
    
    \answer{Yes but two tick delay not one tick and takes up more space.}
}

\questions {
    \item What combination of inputs makes the lamp switch on?
    
    \answer{All inputs must be off.}
    
    \item What combinations of inputs makes the lamp switch off?
    
    \answer{One or more inputs being on.}
    
    \item If you wire all three inputs together, then put them into the block, is it the same thing? 
    
    \answer{Yes. Wire together is an OR gate. The block and torch forms a NOT gate. Combined as OR followed by NOT, they make a NOR gate.}
}

\questions {
    \item What is the output of a NOT gate if its input is on?
    
    \answer{Off.}
    
    \item What is the output of a NOR gate, if it has three inputs of which two are on and one is off?
    
    \answer{Off.}
}


\section{Using NOR gates}

\subsection{AND gates}

\questions {
    \item Which inputs to an AND gate have to be on to make the output (i.e. the lamp) switch on?
    
    \answer{Both inputs must be on.}
    
    \item Which inputs to an AND gate can be off to make the output (i.e. the lamp) switch off?
    
    \answer{Either or both can be off.}
}

\subsection{Locked doors}

\questions {
    \item What is the delay between switching the levers and the door opening and closing?
    
    \answer{(May depend on what circuit they've actually built). For the intended circuit: Min=5 ticks, Max=6 ticks}
    
    \item Can you work out what is causing this delay?
    
    \answer{Delay is caused by torches on blocks i.e. the NOT/NOR gates and the door has a 1 tick switching delay}
    
    \item Here's a tricky one for you: Can you make the circuit smaller?
    
    \answer{Yes. Can be compacted to one NOT gate per \enquote{on} lever and one combining NOR gate (no need for repeater). Can be made either 3 blocks wide, 1 high or 3 blocks high and 1 wide.}
    
    \item Lastly: Does making the circuit smaller reduce the open/close delay?
    
    \answer{Yes. Min=1 gate between levers and door=2 tick delay, Max=2 gates between levers and door=3 tick delay - twice as fast!}
}

\end{document}